\documentclass{article}
\usepackage{garudstyle}
\title{\begin{center}
    Oregon ARML PoTDs - Spring 2024
\end{center}
\begin{center}
    \textbf{PoTD Problem 6}
\end{center}} 
\author{\textbf{Garud Shah}}
\date{\textit{April 6th 2024}}
\begin{document}
\maketitle
\newpage
\tableofcontents
\newpage
\section{Problem}
Let $n \ge 3$ be a positive integer. Consider the polygon formed by the $n$th roots of unity. If
complex $z, w$ are in this polygon, prove $zw$ is too.
\section{Modulus}
Put $z$ and $w$ in exponential form. Let $w = r_w \cdot e^{i \theta_w}$ and 
$z = r_z \cdot e^{i \theta_z}$. Now, notice that the polygon is symmetric with rotation symmetry
of angle $\dfrac{2\pi}{n}$. Thus, we only consider $\theta \pmod{ \dfrac{2 \pi}{n}}$.
\begin{center}
    \begin{asy}
        import olympiad;
        import cse5;
        unitsize(0.5inch);
        int n = 10;
        for (int i=0; i<n; ++i) {
            pair dir1 = dir(360 * i / n) * 5;
            pair dir2 = dir(360 * (i + 1) / n) * 5;
            draw(dir1 -- dir2); 
        }
        drawline((0, 1), (0, -1));
        drawline((1, 0), (-1, 0));
        dot(dir(60)*0.7*5);
        label(dir(60)*0.7*5, "$w = r_w \cdot e^{i \theta_w}$", SE);
    \end{asy}
\end{center}
\section{Law Of Sines}
Noti
\section{Inequalities I}
teete
\section{Inequalities II}
etetet
\end{document}
