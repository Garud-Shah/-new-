\documentclass{article}
\usepackage{garudstyle}
\title{\begin{center}
    Oregon ARML PoTDs - Spring 2024
\end{center}
\begin{center}
    \textbf{PoTD Problem 6}
\end{center}} 
\author{\textbf{Garud Shah}}
\date{\textit{April 6th 2024}}
\begin{document}
\maketitle
\newpage
\tableofcontents
\newpage
\section{Problem}
For what natural numbers $n > 2$ is $\phi (n)$ prime?
\section{Coprime}
Remember: $\phi(n)$ is multiplicative. So, if $p$ is a prime greater than 3, and $k$ divides $p$:
\begin{align}
    & \phi(pn) \mid \phi(k)  \\
    & \phi(pn) = (p - 1) \cdot \phi(n),
\end{align}
so $\phi(n) = 1$, and $n = 1$ or $2$! So, only numbers of the form $2^n$, or $p$, or $2p$ with $p$ prime could be valid. 
\newpage
\section{Testing}
If $p$ is a prime greater than $5$,
since $p$ is odd, $p-1$ is not prime, so $\phi(p)$ and $\phi(2p)$ are not prime. For $2^n$, we have $\phi(2^n) = 2^{n-1}$ which isn't prime
for $n \ge 3$. So, $p = 3$ for $p$ and $2p$ and $k = 2$ for $2^k$ are the only contenders left. \newline
These all work:
\begin{center}
\begin{tabular}{|c|c|}
    \hline
    x & $\phi(x)$ \\
    \hline
    3 & 2 \\
    4 & 2 \\
    6 & 2 \\
    \hline
\end{tabular}
\end{center}
\end{document}