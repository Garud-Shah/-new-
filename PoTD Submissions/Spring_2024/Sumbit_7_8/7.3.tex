\documentclass{article}
\usepackage{garudstyle}
\title{\begin{center}
    Oregon ARML PoTDs - Spring 2024
\end{center}
\begin{center}
    \textbf{PoTD Season 7, Question 3}
\end{center}} 
\author{\textbf{Garud Shah}}
\date{\textit{May 9th 2024}}
\begin{document}
\maketitle
\newpage
\section{Abundancy Of Multiples Of Abundant Numbers}
\textbf{Definition 1.1. }Define the divisor set of $n$ to be $\mathcal{D}(n)$. \\~\\
\textbf{Definition 1.2. }Let $A = kS$ for a set $S$ be, if the elements of $S$ are $S_i$:
\begin{align*}
kS_i = A_i.
\end{align*} \\~\\
\textbf{Definition 1.3. }Let $s(S)$ for a set $S$ be the sum of all elements of $S$. \\~\\
\textbf{Lemma 1.4L. }$k\mathcal{D}(n) \cup 1 \subset \mathcal{D}(kn)$ when $k > 1$. \\
\textbf{Proof 1.4P. }Notice that $k\mathcal{D}(n)$ is all $k \cdot f$ where $f$ is a factor of $n$, now let $f = \dfrac{n}{a}$ for integer $a$.
Then 
\begin{align*}
    k \cdot f = \dfrac{nk}{a}, 
\end{align*}
so $kf \in \mathcal{D}(kn).$ Also since $1$ is the trivial divisor it is in $\mathcal{D}(kn)$. \\~\\
\textbf{Lemma 1.5L. }$s(A \cup B) = s(A) + s(B)$ if $A$, $B$ disjoint. \\
\textbf{Proof 1.5P. }This is a corrolary of the dditive principle.  \\~\\
\textbf{Lemma 1.6L. }$A \subset B \implies s(A) \le s(B)$ if $A$ and $B$ only contain positive integers. \\
\textbf{Proof 1.6P. }Notice that $s(B) - s(A) = s(B \setminus A) - s(A \setminus B)$. \\
Since $A \subset B$, $A \setminus B = \emptyset$, 
\begin{align*}
    s(B) - s(A) = s(B \setminus A) - s(\emptyset).
\end{align*} 
Since empty sums are zero, we have $s(\emptyset) = 0$,
so $s(B) - s(A) = s(B \setminus A)$. Since $A$ and $B$ only contain positive integers, $s(B \setminus A) \ge 0$. So, $s(B) - s(A) \ge 0$, and
$s(B) \ge s(A)$. \\~\\
\textbf{Lemma 1.7L. }$A \subset B \implies s(B) \ge s(A)$ if $A$ and $B$ only contain positive integers. \\
\textbf{Proof 1.7P. }Notice that $s(B) - s(A) = s(B \cup A \setminus A) - s(A \setminus B)$ by Lemma 1.5. \newline
Since $A \subset B$, $B \cup A = B$, and $A \setminus B = \emptyset$, 
\begin{align*}
    s(B) - s(A) = s(B \setminus A) - s(\emptyset).
\end{align*} 
Since empty sums are zero, we have $s(\emptyset) = 0$,
so $s(B) - s(A) = s(B \setminus A)$. Since $A$ and $B$ only contain positive integers, $s(B \setminus A) \ge 0$. So, $s(B) - s(A) \ge 0$, and
$s(B) \ge s(A)$. \\~\\~\\
\textbf{Lemma 1.8L. }$s(kS) = ks(S)$. \\
\textbf{Proof 1.8P. }We sum by element and this is equivilant to moving a constant out of a summation. \\~\\
\textbf{Lemma 1.9L. (Main Theorem of Section)} If $p$ is perfect or abundant, and $k > 1$, $pk$ is abundant. \\
\textbf{Proof 1.9P. }The definition of abundancy is equivilant to saying that $s(\mathcal{D}(n)) \ge 2n$. \\~\\

Plug in $kn$. We get that, by Lemma 1.6, that $s(\mathcal{D}(kn)) \ge s(A)$ for some $A \subset \mathcal{D}(kn)$. \\~\\
By Lemma 1.4, we set $A = k\mathcal{D}(n) \cup 1 \subset \mathcal{D}(kn)$. \\~\\
Since $1$ and $k\mathcal{D}(n)$ are disjoint, by Lemmas 1.5 and 1.8, we have that $s(\mathcal{D}(kn)) \ge ks(\mathcal{D}(n)) + 1$. By our initial 
assumption, $s(\mathcal{D}(n)) \ge 2n$, so $s(\mathcal{D}(kn)) \ge ks(\mathcal{D}(n)) + 1 \ge 2kn + 1$ so $s(\mathcal{D}(kn)) > 2kn$. So, $kn$ is abundant,
and the lemma is proved.
\newpage 
\section{Finding Perfect Numbers Modulo 6}
\textbf{Lemma 2.1L.} If there exist abundant numbers modulo 6 with residue 1, 2, 3, 4, and 5, we are finished. \\
\textbf{Proof 2.1P.} Notice that 6 is perfect. So, by Lemma 1.9, we can add multiples of 6 for our result, and $12$ plus any other multiple of $6$ gives $0$ mod 6.
\\~\\
Notice that \textbf{28} is perfect, so by Lemma 1.9L, we can get 56 and 112 abundant, meaning we only need residues 1, 3, 5. \\
Now, \textbf{945} is abundant, so we have residue 3 down as well. \\~\\
\textbf{Definition 2.2.} The \textit{abundancy index} $a(n)$ is $\dfrac{s(\mathcal{D}(n))}{n}$. \\~\\
Now, if we only include factors of $5, 7, 11, \cdots$ we will get an abundant number. We notice that by the formula for $\sigma(n) = s(\mathcal{D}(n))$, we
can break this up into primes. Now, notice that the maximum we can get for the term of a single prime that is:
\begin{align}
&\lim_{k \rightarrow \infty}\dfrac{1 + p + p^2 + \cdots + p^k}{p^k} \\
&= \lim_{k \rightarrow \infty} 1 + \dfrac{1}{p} + \dfrac{1}{p^2} + \cdots + \dfrac{1}{p^k} \\
\end{align}
Now, multiply this for all primes to get an inf-reachable maximum for abundancy index. Notice that we get the harmonic series, which diverges, 
and if we divide by just the $p=2,3$ terms it still diverges. 
So, plug in an arbitrary number of primes, use arbitrarily large powers, and we clear one of residue 1 or 5. Now multiply by 5, and the other residue of 1 or 5
is done by mods. We are done.

\end{document}