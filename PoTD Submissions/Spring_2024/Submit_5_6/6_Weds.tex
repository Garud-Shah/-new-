\documentclass{article}
\usepackage{garudstyle}
\title{\begin{center}
    Oregon ARML PoTDs - Spring 2024
\end{center}
\begin{center}
    \textbf{PoTD Problem 39}
\end{center}} 
\author{\textbf{Garud Shah}}
\date{\textit{May 9th 2024}}
\begin{document}
\maketitle
$\text{ }$ \newline
Notice that 
\begin{align} 
    2 \cos \left(\dfrac{\pi}{4}\right) &= \sqrt{2} \\
    \cos \left( x - \dfrac{\pi}{4} \right) &\le 1,
\end{align}
so 
\begin{align}
    2 \cos \left(\dfrac{\pi}{4}\right) \cos \left(x - \dfrac{\pi}{4} \right) \le \sqrt{2}.
\end{align} 
So, by product-to-sum, we have 
\begin{align}
    \cos x + \cos \left(x - \dfrac{\pi}{2} \right) \le \sqrt{2}.
\end{align} 
So, since $\dfrac{\pi}{2} > \sqrt{2}$ and $\sin x = \cos \left( x - \dfrac{\pi}{2} \right)$,
\begin{align}
    \dfrac{\pi}{2} > \cos x + \sin x.
\end{align} 
So, $\sin x > \dfrac{\pi}{2} - \cos x$, and now take cosines, giving 
\begin{align}
    \boxed{\cos \sin x > \sin \cos x}.
\end{align}
\end{document}