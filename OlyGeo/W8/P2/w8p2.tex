\documentclass{article}
\usepackage{garudstyle}
\begin{document}
\begin{center}
\begin{asy}
    import olympiad;
    import cse5;
    size(5inch);
    pair A, B, C;
    A = (0, 0);
    B = (5, 5);
    C = (6, 0);
    draw(A -- B -- C -- cycle);
    pair I = incenter(A, B, C);
    pair Aprime = foot(I, B, C);
    pair Bprime = foot(I, C, A);
    pair Cprime = foot(I, A, B);
    pair P = foot(I, A, Aprime);
    draw(I -- Aprime);
    draw(I -- Bprime);
    draw(I -- Cprime);
    draw(I -- P);
    draw(incircle(A, B, C));
    draw(circumcircle(A, Bprime, Cprime));
    draw(A -- Aprime);
    draw(B -- P -- C -- cycle);
    dot(A); dot(B); dot(C);
    dot(Aprime); dot(Bprime); dot(Cprime);
    dot(I); dot(P);
    label("$A$", A, SW);
    label("$A'$", Aprime, NE);
    label("$B'$", Bprime, S);
    label("$C'$", Cprime, NW);
    label("$B$", B, W);
    label("$C$", C, SE);
    label("$I$", I, NW);
    label("$P$", P, S);
\end{asy} 
\end{center}
% Notice that we need
% \begin{align}
%     \dfrac{BP}{BA'} = \dfrac{CP}{CA'}
% \end{align}
% by Angle Bisector Theorem. \newline
Define $B'$, $C'$ similarly to $A'$.
\end{document}